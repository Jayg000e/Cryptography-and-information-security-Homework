\documentclass{ctexart}
\usepackage{ctex}
\usepackage{amsfonts}
\usepackage{amssymb}
\usepackage{pythonhighlight}
\title{密码学与信息安全第二次作业}
\author{郭嘉\quad17345019\quad 数学与应用数学}
\date{\today}
\begin{document}
	\maketitle
	1.(第二章第1题)
	
	要纠正两位错误,我们只需要这八个码的最短距离至少为5
	
	下面的八个码字最短距离为6显然符合这个要求。
	
	(111000000000000000000000)
	
	(000111000000000000000000)
	
	(000000111000000000000000)
	
	(000000000111000000000000)
	
	(000000000000111000000000)
	
	(000000000000000111000000)
	
	(000000000000000000111000)
	
	(000000000000000000000111)
	
	2.(第二章第4题)
	
	不妨设距离为3的两个码字为
	
	a=00000和b=11100
	
	只需要证明在同构意义下存在唯一的两个码字c,d使得a,b,c,d组成二元[5,4,3]码
	
	首先分析c的末两位:由抽屉原理,存在a,b中的一个,使得他们的前三位有两位与c相同,这就说明了c的末两位必须为11,否则最短距离小于3。
	
	对d也同理,它得末两位也是11
	
	再考虑c,d的前三位,由于还需要至少1位与a,b不同,所以它们的前三位不能全是1或全是0,故有两个0或两个1。
	
	然而注意c,d间的距离也得是3,所以他们前三位之和肯定是111,比如说100+011 010+101 001+110
	
	但是这三种方案在同构意义下是一样的,所以我们就得到了唯一性。
	
	并给出了这四个码字
	
    00000 11100 10011 01111
    
    得证
    
     
	3.(第二章第5题)
	
	如果存在二元[n,K,d]码,我们找到距离为d的两个码字,
	比如说他们的第i位不等,那我们把所有码字的第i位去掉,那么就得到了一个二元[n-1,K,d-1]码
	
	如果存在二元[n-1,K,d-1]码,我们找到距离为d-1的两个码字,这两个码字末位后再加1位,分别加0,1。对于其它码字,我们在他们末位也加一位,但是随便加就行。那么我们就得到了一个二元[n,K,d]码
	
	得证
	
	4.(第二章第7题)
	
	(1)前面两个参数分别是2n和$K_1K_2$是显然的,我们只需要说明最小距离是min ( $d_1,d_2$ )
	
	对于两个不同的码字( $c_1,c_2$ )和( $c_1',c_2'$ )
	
	如果$c_1=c_1'$ $c_2 \neq c_2'$ 则两者距离最小值为$d_2$
	
	如果$c_1 \neq c_1'$ $c_2 = c_2'$ 则两者距离最小值为$d_1$
	
	如果$c_1 \neq c_1'$ $c_2 \neq c_2'$ 则两者距离最小值为$d_1+d_2$
	
	综上所述,最小距离是min ( $d_1,d_2$ )
	
	(2)前面两个参数分别是2n和$K_1K_2$是显然的,我们只需要说明最小距离是min ( $2d_1,d_2$ )
	
	如果$c_1=c_1'$ $c_2 \neq c_2'$ 则两者距离最小值为$2d_1$
	
	如果$c_1 \neq c_1'$ $c_2 = c_2'$ 则两者距离最小值为$d_2$
	
	如果$c_1 \neq c_1'$ $c_2 \neq c_2'$ 首先考察前n位的最小距离显然为$d_1$
	
	然后考察后n位,即d($c_1 +c_2,c_1' +c_2'$)
	
	由三角不等式
	
	d($c_1 +c_2,c_1' +c_2'$)+d($c_1' +c_2',c_1 +c_2'$)
	$\geq$ d($c_1 +c_2,c_1 +c_2'$)
	
	即d($c_1 +c_2,c_1' +c_2'$)$\geq$$d_2-d_1$
	
	故该种情况距离至少是$d_1+d_2-d_1=d_2$
	
	综上所述最小距离是min ( $2d_1,d_2$ )
	
	得证
	
\end{document}