\documentclass{ctexart}
\usepackage{ctex}
\usepackage{amsfonts}
\usepackage{amssymb}
\usepackage{amsmath}
\usepackage{pythonhighlight}
\title{密码学与信息安全第三次作业}
\author{郭嘉\quad17345019\quad 数学与应用数学}
\date{\today}
\begin{document}
	\maketitle
	1.(第三章第1题)
	
	该码若为线性码,必须为$\mathbb{F}$$_{2}^{8}$的k维子空间,然而K=20并不是2的幂次,故该码不可能为线性码。
	
	2.(第三章第4题)
	
	分两个方向证明。
	
	若该码为等距码,则对于$\forall$$c_1$,$c_2$$\in$C,有w($c_1$)=d($c_1$,0)=d($c_2$,0)=w($c_2$),故为等重码
	
	若该码为等重码,则对于$\forall$$c_1$,$c_2$,$c_1'$,$c_2'$$\in$C,有d($c_1$,$c_2$)=w($c_1$-$c_2$)=w($c_1'$-$c_2'$)=d($c_1'$,$c_2'$),故为等距码
	
	得证
    
     
	3.(第三章第7题)
	
	前两问在课上讲过相同的例题,这里仅作简略说明
	
	(1)n=$\frac{3^2-1}{3-1}$=4
	
	k=4-2=2
	
	d=3
	
	(2)$$H=
	\begin{bmatrix}
	1 & 0 & 1 & 1\\
	0 & 1 & 1 & 2 
	\end{bmatrix} 
	$$
	
	这里容易验证这四个列向量任意两个线性无关
	
	写出校验矩阵H后容易写出生成矩阵G:
	
	$$G=
	\begin{bmatrix}
	2 & 2 & 1 & 0\\
	2 & 1 & 0 & 1 
	\end{bmatrix} 
	$$
	
	(3)$\alpha^T=Hv^T=
	\begin{bmatrix}
	2 \\
	2  
	\end{bmatrix} 
	=2\begin{bmatrix}
	1 \\
	1  
	\end{bmatrix} $
	
	即$\epsilon=\begin{bmatrix}
	0 & 0 & 2 & 0
	\end{bmatrix} $
	
	故$u=v-\epsilon=\begin{bmatrix}
	2 & 0 & 2 & 2
	\end{bmatrix}$
	
	4.(第三章第8题)
	
	根据本章习题4,我们只需要证明重量部分。
	
	根据定义,Ham(r,2)对偶码的生成矩阵G是$r\times(2^r-1)$的矩阵,且它的列向量是
	$\mathbb{F}$$_{2}^{r}$中除了0向量外其它的所有元素,注意到这点之后我们就知道
	G的每个行向量,都有$2^r/2=2^{r-1}$个1和$2^r/2-1=2^{r-1}-1$个0.故$\forall x \in \{$   $x|x \in \mathbb{F}_{2}^{r},x$仅有一个分量为1,剩余分量都为0 $\}$ 有xG为G中其中一个行向量,故xG重量为$2^{r-1}$。但现在我们需要证明的是
	对$\forall x \in \{x|x \in \mathbb{F}_{2}^{r},x\neq0\}$都有xG重量为$2^{r-1}$,即若干个G中的行向量之和的重量为$2^{r-1}$。
	
	下面采用计数的方法说明G中两个行向量之和重量为$2^{r-1}$,考虑这两个行向量对应分量元素只能为$\begin{bmatrix}
		0 \\
		0 
	\end{bmatrix}$ $\begin{bmatrix}
	0 \\
	1 
\end{bmatrix}$ $\begin{bmatrix}
1 \\
0 
\end{bmatrix}$ $\begin{bmatrix}
1\\
1 
\end{bmatrix}$这四种情况出现的次数分别为$2^r/4-1$,$2^r/4$,$2^r/4$,$2^r/4$

故和为1的中间两种情况加起来共出现$2^{r-1}$次,这就说明了论断。

对于G中m个行向量之和,m<=r,利用同样的计数方法,重量为$2^{r-m}\sum_{i=0}^{\lfloor m/2 \rfloor}\binom{r}{2i+1}=2^{r-1}$

得证
   
\end{document}