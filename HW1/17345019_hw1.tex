\documentclass{ctexart}
\usepackage{ctex}
\usepackage{amsfonts}
\usepackage{amssymb}
\usepackage{pythonhighlight}
\title{密码学与信息安全第一次作业}
\author{郭嘉\quad17345019\quad 数学与应用数学}
\date{\today}
\begin{document}
	\maketitle
	1.(第一章第3题)
	
	只需证明该环中任一元素$\alpha$都存在逆元。
	
	我们取$\alpha$、$\alpha^2$、$\alpha^3$、$\alpha^3$...
	
	注意到该环是有限环,则必存在a,b$\in \mathbb{Z+}$ 使得$\alpha^a=\alpha^b$(a>b)
	
	交换整环满足消去律,即$\alpha^{a-b}=1$
	
	即$\alpha$的逆元为$\alpha^{a-b-1}$
	
	得证
	
	2.(第一章第4题)
	
	
	由于$\underbrace{1+1+...+1}_{ch(\mathbb{F})}=0$ 
	
	故对于$ {\forall}\alpha\in\mathbb{F}$,$(\underbrace{1+1+...+1}_{ch(\mathbb{F})})*\alpha=0$ 
	
	即$\underbrace{\alpha+\alpha+...+\alpha}_{ch(\mathbb{F})}=0$ 
	
	反之,若$\underbrace{\alpha+\alpha+...+\alpha}_{n}=0$且$\alpha\neq0$
	
	则n不可能小于ch($\mathbb{F}$)
	
	否则$(\underbrace{1+1+...+1}_{n})*\alpha=0$
	
	由于$\alpha\neq0$,故$\underbrace{1+1+...+1}_{n}=0$
	
	故ch($\mathbb{F}$)|n,这与n<ch($\mathbb{F}$)矛盾
	
	综上所述,每个非零元素的阶都是n
	
	得证
	
	
	
	3.(第一章第5题)
	
	由有限域的结构定理,所有四阶有限域都同构。
	
	特别的由于容易验证$x^2+x+1$是$\mathbb{F}$$_{2}$上的二次既约多项式
	
	故有$\mathbb{F}$$\cong$$\mathbb{F}$$_{2}$$/(x^2+x+1)$
	
	故$\mathbb{F}$中的4个元素分别为0,1,a,a+1,其中$a^2+a+1=0$
	
    (1)如上所述$\mathbb{F}$$\cong$$\mathbb{F}$$_{2}$$/(x^2+x+1)$,故1+1=0,即ch($\mathbb{F}$)=2
    
    (2)如上所述故$\mathbb{F}$中的4个元素分别为0,1,a,a+1,其中$a^2+a+1=0$,
    
    故$a^2=-a-1=a+1$和$(a+1)^2=a^2+1=(a+1)+1$
    
    得证
    
    4.(第一章第12题)
    
    (1)由于 |$\mathbb{F}$$_{8}^{*}$|=7
    
    故其中元素的阶要么为7,要么为1,若为本原元即为7,否则为1
    
    但$\varphi$(7)=6,故除了乘法单位元1外,乘法群中其它元素阶都为7
    
    (2)利用$\alpha^3+\alpha+1=0$
    
    有$\alpha^3+\alpha^5=(\alpha+1)+(\alpha^3+\alpha)=\alpha^3+1=\alpha$
    
    利用$(1+\alpha)^7=1$
    
   	有$(\alpha+1)^{-1}=(\alpha+1)^{6}=\alpha^6+6\alpha^5+15\alpha^4+20\alpha^3+15\alpha^2+6\alpha+1=\alpha^6+\alpha^4+\alpha^2+1=(\alpha+1)^2+(\alpha^2+\alpha)+\alpha^2+1=\alpha^2+\alpha$
   	
   	5.(第一章第13题)
   	
   	(1)本章例题中已验证了$x^4+3x^2+1$与$x^4+x^2+1$都是$\mathbb{F}$$_{2}$上除了$x^4+x^3+x^2+x+1$外仅有的既约多项式。又$\varphi$(16-1)=8,即本原元有八个,故这八个本原元就是$x^4+3x^2+1=0$与$x^4+x^2+1=0$的所有根。
   	
   	容易验证$\alpha+1$是$x^4+3x^2+1=0$的一个根,那么$(\alpha+1)^2=\alpha^2 + 1$,$(\alpha+1)^4=\alpha^3 + \alpha^2 + \alpha$,$(\alpha+1)^8=\alpha^3+1$是其余的三个根
   	
   	容易验证$\alpha^3+\alpha$是$x^4+x^2+1=0$的一个根,那么$(\alpha^3+\alpha)^2=\alpha^2 + \alpha$,$(\alpha^3+\alpha)^4=\alpha^3 + \alpha + 1$,$(\alpha^3+\alpha)^8=\alpha^2+\alpha+1$是其余的三个根
   	
   	上面所说的八个根就是$\mathbb{F}$$_{16}$的全部本原元
   	
   	(2)如(1)中所述$\alpha+1$的极小多项式是$x^4+3x^2+1=0$
   	
   	6.(第一章第15题)
   	
   	$\mathbb{F}$$_{2}$上五次既约多项式个数为$1/5\sum_{d|5}\mu(d)2^{5/d}=6$
   	
   	$\mathbb{F}$$_{2}$上五次既约多项式个数为$1/5\varphi(32-1)=6$
   	
   	故只需要找$\mathbb{F}$$_{2}$上的6个五次既约多项式
   	
   	如例题中的推理,五次项和常数项是必须的,一次项到四次项只能有奇数个,即1个或3个,但这里总共有8个多项式,我们需要排除两个
   	
   	$(x^3+x^2+1)(x^2+x+1)=x^5+x+1$和$(x^3+x+1)(x^2+x+1)=x^5+x^4+1$就是我们所排除的,于是我们可以列出余下6个就是所求既约且本原的多项式
   	
   	分别为$x^5+x^2+1$,$x^5+x^3+1$,$x^5+x^3+x^2+x+1$,$x^5+x^4+x^2+x+1$,$x^5+x^4+x^3+x+1$,$x^5+x^4+x^3+x^2+1$
   	
   	7.补充题1
   	
   	$z^{81}-z$在$\mathbb{F}$$_{3}$上分解得到的所有四次因式就是$\mathbb{F}$$_{3}$上所有的四次既约多项式
   	
   	利用sagemath:
   	
   	\begin{python}
   		R1.<z> = PolynomialRing(GF(3))
   		for i in (z^81-z).factor():
   			print(i[0])
   	\end{python}
   
   输出结果为:
   
   $z$
   
   $z + 1$
   
   $z + 2$
   
   $z^2 + 1$
   
   $z^2 + z + 2$
   
   $z^2 + 2*z + 2$
   
   $z^4 + z + 2$
   
   $z^4 + 2*z + 2$
   
   $z^4 + z^2 + 2$
   
   $z^4 + z^2 + z + 1$
   
   $z^4 + z^2 + 2*z + 1$
   
   $z^4 + 2*z^2 + 2$
   
   $z^4 + z^3 + 2$
   
   $z^4 + z^3 + 2*z + 1$
   
   $z^4 + z^3 + z^2 + 1$
   
   $z^4 + z^3 + z^2 + z + 1$
   
   $z^4 + z^3 + z^2 + 2*z + 2$
   
   $z^4 + z^3 + 2*z^2 + 2*z + 2$
   
   $z^4 + 2*z^3 + 2$
   
   $z^4 + 2*z^3 + z + 1$
   
   $z^4 + 2*z^3 + z^2 + 1$
   
   $z^4 + 2*z^3 + z^2 + z + 2$
   
   $z^4 + 2*z^3 + z^2 + 2*z + 1$
   
   $z^4 + 2*z^3 + 2*z^2 + z + 2$
   
   其中最高次项为4的多项式就是所求既约多项式
   
   对于其中的四次多项式P(x),取$\mathbb{F}$$_{3}$[x]/(P(x))
   
   若对于$\forall$ $1<n<80$ 都有$x^n\neq1$
   
   那么P(x)就是本原多项式
   
   利用sagemath:

	\begin{python}
   for i in range(6,len(list((z^81-z).factor()))):
   		poly=(z^81-z).factor()[i][0]
   		R2.<x> = R1.quotient((poly)*R1)
   		flag=1
   		for j in range(1,80):
   			if x^(j)==1:
   			flag=0
   			break;
   		if flag==1:
   			print(poly)
	\end{python}
   
   输出结果:
   
   $z^4 + z + 2$
   
   $z^4 + 2*z + 2$
   
   $z^4 + z^3 + 2$
   
   $z^4 + z^3 + z^2 + 2*z + 2$
   
   $z^4 + z^3 + 2*z^2 + 2*z + 2$
   
   $z^4 + 2*z^3 + 2$
   
   $z^4 + 2*z^3 + z^2 + z + 2$
   
   $z^4 + 2*z^3 + 2*z^2 + z + 2$
   
   这就是我们需要的本原多项式
   
   8.补充题2
   
   右推左:
   
   若$F(x)=\sum_{n \leq x}\mu(n)G(\frac{x}{n})$
   
   则$\sum_{n \leq x}F(\frac{x}{n})=\sum_{n \leq x}\sum_{nm \leq x}\mu(m)G(\frac{x}{mn})=\sum_{a \leq x}G(\frac{x}{a})\sum_{m|a}\mu(m)$
   
   当a=1时第二个和式取1,对其它a,第二个和式为0
   
   故该式最终等于G(x),这就完成了右推左。
   
   左推右:
   
   若$G(x)=\sum_{n \leq x}F(\frac{x}{n})$
   
   则$\sum_{n \leq x}\mu(n)G(\frac{x}{n})=\sum_{n \leq x}\mu(n)\sum_{nm \leq x}F(\frac{x}{mn})=\sum_{n \leq x}\sum_{nm \leq x}\mu(m)F(\frac{x}{mn})=\sum_{a \leq x}F(\frac{x}{a})\sum_{m|a}\mu(m)$
   
   当a=1时第二个和式取1,对其它a,第二个和式为0
   
   故该式最终等于F(x),这就完成了左推右。
   
   得证
   
   	
   	
   	
   	
   	
   	
   	
   	
   	
   	
   	
   	
   	
  
\end{document}