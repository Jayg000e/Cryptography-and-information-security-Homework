\documentclass{ctexart}
\usepackage{ctex}
\usepackage{amsfonts}
\usepackage{amssymb}
\usepackage{amsmath}
\usepackage{pythonhighlight}
\title{密码学与信息安全第四次作业}
\author{郭嘉\quad17345019\quad 数学与应用数学}
\date{\today}
\begin{document}
	\maketitle
	1.(第四章第1题)
	
	对$x^4-1$分解得$x^4-1=(x+1)^4$
	
	故下面容易对三种不同情况直接写出结果
    
    (1)当生成多项式$g(x)=x+1$
    
    故生成矩阵$$G=
    \begin{bmatrix}
    1 & 1 & 0 & 0\\
    0 & 1 & 1 & 0\\
    0 & 0 & 1 & 1
    \end{bmatrix} 
    $$
    
    进而得到C=
    $\left\lbrace xG|x\in \Pi_{2}^{3}\right\rbrace$ 
    $=\left\lbrace(0000),(1 1 0 0),(0 1 1 0),(0 0 1 1),(1 0 0 1),(1 0 1 0),(0 1 0 1),(1111)\right\rbrace$
    
    也可以得到校验多项式$h(x)=(x^4-1)/g(x)=x^3+x^2+x+1$
    
    故校验矩阵为$$H=
    \begin{bmatrix}
    1 & 1 & 1 & 1
    
    \end{bmatrix} 
    $$
    
    (2)当生成多项式$g(x)=(x+1)^2=x^2+1$
    
    故生成矩阵$$G=
    \begin{bmatrix}
    1 & 0 & 1 & 0\\
    0 & 1 & 0 & 1
    \end{bmatrix} 
    $$
    
    进而得到C=
    $\left\lbrace xG|x\in \Pi_{2}^{2}\right\rbrace$ 
    $=\left\lbrace(0000),(1 0 1 0),(0 1 0 1),(1111)\right\rbrace$
    
    也可以得到校验多项式$h(x)=(x^4-1)/g(x)=x^2+1$
    
    故校验矩阵为$$H=
    \begin{bmatrix}
    1 & 0 & 1 & 0\\
    0 & 1 & 0 & 1
    \end{bmatrix} 
    $$
    
    (3)当生成多项式$g(x)=(x+1)^3=x^3+x^2+x+1$
    
    故生成矩阵$$G=
    \begin{bmatrix}
    1 & 1 & 1 & 1
    \end{bmatrix} 
    $$
    
    进而得到C=
    $\left\lbrace xG|x\in \Pi_{2}^{1}\right\rbrace$ 
    $=\left\lbrace(0000),(1111)\right\rbrace$
    
    也可以得到校验多项式$h(x)=(x^4-1)/g(x)=x+1$
    
    故校验矩阵为$$H=
    \begin{bmatrix}
    1 & 1 & 0 & 0\\
    0 & 1 & 1 & 0\\
    0 & 0 & 1 & 1
    \end{bmatrix} 
    $$

	
	
	2.(第四章第2题)
	
	必要性:如果$C_{2} \subset C_{1}$,那么由于$g_{2}(x)\in C_{2}$,那么$g_{2}(x)\in C_{1}$,但由于$g_{1}(x)$是$C_{1}$的生成多项式,故其整除$C_{1}$中所有的码字,特别的,有$g_{1}(x)|g_{2}(x)$
	
	充分性:如果$g_{1}(x)|g_{2}(x)$,那么$\forall c(x) \in C_{2}$ 有$g_{2}(x)|c(x)$ ,又由于$g_{1}(x)|g_{2}(x)$因此$g_{1}(x)|c(x)$,这说明
	$c(x) \in C_{1}$
	
	得证
     
	3.(第四章第6题)
	
	由第2题结论$C' \subset C$,因此 C'的最小距离$\geq$C的最小距离,即C'的最小距离$\geq$d
	
	下面只需要说明C'的最小距离不可能为d,由于d为奇数,我们只需要说明C'的最小距离不可能为奇数。又由于C'为线性码,故其最小距离等于最小重量,因此我们只需要说明C'码字的最小重量不可能为奇数
	
	我们注意到$\forall c(x) \in C'$ 都有$g(x)(x-1)|c(x)$。特别地,有$(x-1)|c(x)$即c(1)=0 即c(x)的系数和为0,这就意味着c(x)的重量为偶数,特别地C'码字的最小重量为偶数,再结合上面的讨论就完成了证明。
	
	得证
	
	
	
	4.(第四章第7题)
	
    非系统: $a(x)=x+x^3$
    
    $c(x)=a(x)g(x)=x+x^2+x^3+x^6$即(0111001)
    
    系统:
    $\overline{a(x)}=x^6+x^4=(x^3+1)(x^3+x+1)+(x+1)$
    
    即$r(x)=x+1$ ,$c(x)=\overline{a(x)}-r(x)=x^6+x^4+x+1$即(1100101)
\end{document}