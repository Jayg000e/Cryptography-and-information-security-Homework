\documentclass{ctexart}
\usepackage{ctex}
\usepackage{amsfonts}
\usepackage{amssymb}
\usepackage{amsmath}
\usepackage{pythonhighlight}
\title{密码学与信息安全第五次作业}
\author{郭嘉\quad17345019\quad 数学与应用数学}
\date{\today}
\begin{document}
	\maketitle
	1.(第四章第5题)
	
	为了得到C的校验矩阵,我们用p(x)的根来刻画循环码C=(p(x))
	
	设p(x)在$\mathbb{F}_{2^r}$的一个根为$\beta$
	
	那么其它的r-1个根为$\beta^{2}$,$\beta^{2^2}$...$\beta^{2^{r-1}}$
	
	则容易验证$C=\left\lbrace c(x)=\sum_{i=0}^{2^r-2}c_ix^i \in \Pi_2^{(2^r-1)}|c(\beta)=c(\beta^2)=c(\beta^{2^2})=...=c(\beta^{2^{r-1}})=0 \right\rbrace $
	
	也容易说明C的校验矩阵为$$H_C=
	\begin{bmatrix}
	1 & \beta & ... & \beta^{n-1}\\
	1 & \beta^{2} & ... & \beta^{2(n-1)}\\
	... & ... & ... & ...\\
	1 & \beta^{2^{r-2}} & ... & \beta^{2^{r-2}(n-1)}\\
	1 & \beta^{2^{r-1}} & ... & \beta^{2^{r-1}(n-1)}
	\end{bmatrix} 
	$$
	
	而Ham(r,2)的校验矩阵$H_{Ham}$以$\Pi_2^{r}$中除了零向量以外的向量为列向量

	可以说明对于$\forall c \in \Pi_2^{(2^r-1)}$ $H_{Ham}c^T=0$当且仅当$H_{C}c^T=0$(我在这里的说明遇到了困难,不过这个论断应该是正确的,而且应该可以通过某些技巧验证)
	
	因此这两个码等价
	
	2.(第五章第3题)
	
	容易直接写出
	
	$B_{2}(15,9,\beta)=\left\lbrace c(x)=\sum_{i=0}^{14}c_ix^i \in \Pi_2^{(15)}|c(\beta)=c(\beta^3)=c(\beta^5)=c(\beta^7)=0 \right\rbrace $
	
	其中$\beta$为$x^4+x+1$的一个根
	
	为求生成多项式,只需要求$\beta,\beta^2,\beta^3,\beta^4,\beta^5,\beta^6,\beta^7,\beta^8$在$\Pi_{2}$上的极小多项式。
	
	由于书中例题已经求出其中一些,我们只需要求$\beta^7$在$\Pi_{2}$上的极小多项式,容易得到$m_{7}(x)=(x-\beta^7)(x-\beta^{14})(x-\beta^{28})(x-\beta^{56})=x^4+x^3+1$
	
	故生成多项式$g(x)=m_{1}(x)m_{3}(x)m_{5}(x)m_{7}(x)=(x^4 + x + 1)(x^4 + x^3 + x^2 + x + 1)(x^2 + x + 1)(x^4 + x^3 + 1)=x^{14} + x^{13} + x^{12} + x^{11} + x^{10} + x^9 + x^8 + x^7 + x^6 + x^5 + x^4 + x^3 + x^2 + x + 1$
	

    3.(第五章第5题)
    
    (1)计算校验子
    
    $v(\beta)=\epsilon(\beta)=\beta^{11}$
    
    $v(\beta^2)=\epsilon(\beta^2)=\beta^7$
    
    $v(\beta^3)=\epsilon(\beta^3)=\beta^{13}$
    
    $v(\beta^4)=\epsilon(\beta^4)=\beta^{14}$
    
    $v(\beta^5)=\epsilon(\beta^5)=1$
    
    $v(\beta^6)=\epsilon(\beta^6)=\beta^{11}$
    
    (2)解非齐次线性方程组
    
    $\epsilon(\beta^3)\sigma_1+\epsilon(\beta^2)\sigma_2+\epsilon(\beta) \sigma_3=\epsilon(\beta^4)$
    
     $\epsilon(\beta^4)\sigma_1+\epsilon(\beta^3)\sigma_2+\epsilon(\beta^2) \sigma_3=\epsilon(\beta^5)$
     
      $\epsilon(\beta^5)\sigma_1+\epsilon(\beta^4)\sigma_2+\epsilon(\beta^3) \sigma_3=\epsilon(\beta^6)$
      
     把(1)中所得结果带入上述方程组,并求解
     
     得到$\sigma_1=\beta^11$ $\sigma_2=\beta^2$ $\sigma_3=\beta^3$
     
     故差错位置多项式为$\sigma(z)=1+\sigma_1z+\sigma_2z^2+\sigma_3z^3$
     
     得到其三个根分别为$\beta^{-3},\beta^{-6}, \beta^{-9}$
     
     故差错位置为3,6,9
     
     差错多项式为$\epsilon(x)=x^3+x^6+x^9$
     
     原码字多项式$c(x)=v(x)-\epsilon(x)=1+x^2+x^3+x^6+x^8+x^{13}+x^{14}$
    
	
	
	
	
	
	4.(第五章第6题)
	
	容易直接写出
	
	$S(7,3,\beta)=$ 
	
	$\left\lbrace c_f=(f(1),f(\beta),f(\beta^2),f(\beta^3),f(\beta^4),f(\beta^5),f(\beta^6)) \in \mathbb{F}_8^7|\forall f(x) \in \mathbb{F}_8[x], deg(f(x))\leq k-1\right\rbrace$
	
	其中k=n-$\delta$+1=7-3+1=5
   
   生成矩阵为$$G=
   \begin{bmatrix}
   1 & 1 & 1 & 1& 1& 1& 1\\
   1 & \beta & \beta^2 & \beta^3& \beta^4& \beta^5& \beta^6\\
   1 & \beta^2 & \beta^4 & \beta^6& \beta^1& \beta^3& \beta^5\\
   1 & \beta^3 & \beta^6 & \beta^2& \beta^5& \beta^1& \beta^4\\
   1 & \beta^4 & \beta^1 & \beta^5& \beta^2& \beta^6& \beta^3
   \end{bmatrix} 
   $$
\end{document}